\documentclass{article}

% Language setting
% Replace `english' with e.g. `spanish' to change the document language
\usepackage[french]{babel}
\usepackage[fleqn]{amsmath} % Aligner les équations à gauche


% Set page size and margins
% Replace `letterpaper' with`a4paper' for UK/EU standard size
\usepackage[letterpaper,top=2cm,bottom=2cm,left=3cm,right=3cm,marginparwidth=1.75cm]{geometry}

% Useful packages
\usepackage{amsmath}
\usepackage{graphicx}
\usepackage{subcaption}
\usepackage[colorlinks=true, allcolors=blue]{hyperref}

\title{TD8}
\author{IPESUP - PC }
\date{11 Janvier 2024}

\begin{document}
\maketitle



\section{Rappels de cours}
\begin{enumerate}

    \item \textbf{Définitions:}\\[0.1cm]
    \begin {itemize}
    \item Ligne de courant : Courbe de l'espace qui possède en tout point une tangente parralèle à la vitesse du fluide.\\
    \item Tube de courant: Ensemble des lignes de courant qui passent par un contour fermé.\\
    \item Fluide parfait : Fluide sans viscosité.\\
    \end{itemize}
    \item  \textbf{Conservation de la masse : }\\[0.1cm]
$     \frac{\partial \rho}{\partial t} + div(\rho \vec{v}) = 0$ \\

On a souvent $\rho = cste$ , donc $ div(\vec{v}) = 0 $ . \\[0.1cm]

\\
\item \textbf{Accélération particulaire :}
 \\[0.1cm]

$\frac{Df}{Dt} = \frac{\partial f}{\partial t} + \vec{v} \cdot \vec{grad}(f)$\\

$\vec{v} \cdot \vec{grad}(f)$ est un opérateur : \\

$\vec{v} \cdot \vec{grad}(f)$ = $v_x \frac{\partial f}{\partial x} + v_y \frac{\partial f}{\partial y} + v_z \frac{\partial f}{\partial z}$\\

Pour une variable vectorielle, on a : \\[0.2cm]
$\vec{v} \cdot \vec{grad}(\vec{f})$ = $\vec{v} \cdot \vec{grad} (f_x) \vec{e_x} + \vec{v} \cdot \vec{grad} (f_y) \vec{e_y} + \vec{v} \cdot \vec{grad} (f_z) \vec{e_z} $\\

\item \textbf{Equations dynamiques : }\\[0.1cm]

Equation d'Euler: $\rho \frac{D\vec{v}}{Dt} = -grad(P) + \rho \vec{g} $\\[0.1cm] 
\\
On obtient cette équation en appliquant le PFD à une particule de fluide. 

\item \textbf{ Théorème de Bernoulli : }\\[0.1cm]
Pour un Fluide parfait, en Régime stationnaire, dans un écoulement Incompressible , dans un référentiel Galiléen, et si le fluide est hOmogène (FRIGO),  alors on a (sur une ligne de courant): \\[0.1cm]
$P + \frac{1}{2} \rho v^2 + \rho g z = cste$\\[0.1cm]

Si en plus le fluide est irrotationnel, cette formule est vraie partout ! (pas que sur une ligne de courant)
\end{enumerate}
\section{A quelle vitesse va la voiture ? }

\section{Profil de température}



\end{document}