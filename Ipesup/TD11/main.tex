\documentclass{article}

% Language setting
% Replace `english' with e.g. `spanish' to change the document language
\usepackage[french]{babel}
\usepackage[fleqn]{amsmath} % Aligner les équations à gauche


% Set page size and margins
% Replace `letterpaper' with`a4paper' for UK/EU standard size
\usepackage[letterpaper,top=2cm,bottom=2cm,left=3cm,right=3cm,marginparwidth=1.75cm]{geometry}

% Useful packages

\usepackage{amsmath}
\usepackage{graphicx}
\usepackage{subcaption}
\usepackage[colorlinks=true, allcolors=blue]{hyperref}

\title{TD 11}
\author{IPESUP - PC }
\date{31/01/2024}

\begin{document}
\maketitle



\section{L'écoulement du ketchup}

On considère l’écoulement laminaire, stationnaire, irrotationnel et incompressible du ketchup
assimilé à un fluide visqueux, de masse volumique $\my$ , dans une conduite verticale et cylindrique
sous le seul effet de la pesanteur. On suppose que la pression ne varie pas le long du
tube. On note $Oz$ l’axe du tube. Le champ eulérien des vitesses est, en coordonnées cylindriques
: $\vec{v(M)} = v(r) \vec{u_z}$, où $r$ est la distance du point $M$ à l’axe du tube, et où le vecteur $\vec{u_z}$
est orienté dans le même sens que l’accélération de la pesanteur $_vec{g}$.
On note $\sigma(r)$ la projection sur $\vec{u_z}$ de la force tangentielle surfacique que le fluide situé à
l’intérieur du cylindre de rayon $r$ exerce sur le fluide situé à l’extérieur du cylindre de
rayon $r$. Le ketchup est un fluide non newtonien caractérisé par le comportement rhéologique
suivant :


${\frac{\mathrm{d}v}{\mathrm{d}r}}={\frac{\sigma_{0}-\sigma(r)}{0}\quad{\mathrm{pour}}\;\sigma(r)\geq\sigma_{0}}\\ 
\frac{\mathrm{d}v}{\mathrm{d}r}}=  {0}&{{\mathrm{ pour }\;\sigma(r)\leq\sigma_{0}}\end{array}}$


Les deux paramètres $\eta$ (homogène à une viscosité dynamique) et $\sigma_{0}$ sont deux caractéristiques
du fluide étudié.

\begin{enumerate}
  \item On considère une particule de fluide, comprise entre $z$ et $z + dz$ et entre $r$ et $r + dr$. Établir
  l’équation différentielle du mouvement de cette particule de fluide.
  \item Intégrer cette équation différentielle et donner l’expression de $\sigma(r)$.
  \item En déduire l’expression du champ des vitesses. On sera amené à distinguer deux cas, selon
  que $r$ est supérieur ou inférieur à un rayon caractéristique $r_0$. On fera de plus l’hypothèse que
  $r_0 < R$.\\
  Donner l'allure du champ des vitesses en fonction du rayon $r$. Expliquer pourquoi on parle
  d’écoulement bouchon.
  \item Donner l’expression du champ des vitesses lorsque $r_0 > R$ . Déterminer l’ordre de grandeur
  du rayon minimal d’un tube permettant l’écoulement du ketchup : $\sigma_0$ de l’ordre de 50 Pa et
  $\eta$ de l’ordre de $10^{6}\,\mathrm{Pa}.s\,\mathrm{et}\,\mu=1,4\times10^{3}\,\mathrm{kg.m}^{-3}.$
\end{enumerate}
\section{Propagation d'une onde sonore dans un tuyau déformable}




Un tube horizontal, de longueur infinie, cylindrique, d’axe Ox, contient un fluide de masse
volumique $\mu$ et de compressibilité $\chi_0$.
Le tube est élastique de section variable : $S(x,t)=S_{0}+S_{1}(x,t),$ où $ S_{1}(x,t) << S_{0}.$ On suppose
que l’équation de comportement du tuyau permet de définir sa section comme fonction de la
pression uniquement, sous la forme $S = S(P)$.
On néglige les effets de la pesanteur et de la viscosité. On se place dans le cadre de l’approximation acoustique.

On définit la distensibilité du tube par $:D={\frac{1}{S}}{\frac{\mathrm{d}S}{\mathrm{d}P}}.$

\begin{enumerate}
  \item Effectuer un bilan de matière pour le système ouvert contenu entre les plans d’abscisses
  $x$ et $x+ dx $ entre les instants $t$ et $t+dt$. Linéariser cette équation et en déduire une relation entre ${\frac{\partial p_{1}}{\partial t}}(x,t)$ et ${\frac{\partial v}{\partial t}}(x,t)$
  faisant intervenir le coefficient de compressibilité au repos $\chi_0$ et
 la distensibilité $D0$ au repos.
 \item En déduire l’équation de propagation des ondes sonores dans le tube et montrer que leur
 célérité est donnée par :

 $c={\frac{1}{\sqrt{\mu_{0}(\chi_{0}+D_{0})}}}$

  \item Application : la masse volumique et la compressibilité du sang et de l'eau sont comparables.
  Dans les conditions de l’expérience, la célérité du son dans l’eau vaut $c_{eau} = 1,4 \times 10^3 m.s^{-1}$. 
  On donne $D_{0m}=1.0\times10^{-11} P_a^{-1}$ pour un tuyau métallique et $D_{0v}=4.0\times10^{-5} P_a^{-1}$
  pour un vaisseau sanguin.
  Calculer c dans les deux cas. Commenter.

  \item On place en $x = 0$ une pompe assurant un débit massique $D_m(t)$. En supposant qu’aucune
  onde ne provient de l’infini, déterminer l’expression de la vitesse $v(x,t)$ et de la surpression
  $p_1(x,t)$ du fluide sur le demi-axe $x > 0$ à l’aide de la fonction $D_m (t)$ .

\end{enumerate}


\section{ Mines 2 2022} 

Questions 16 à 25. 
 

\end{document}


