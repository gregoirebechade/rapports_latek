\documentclass{article}

% Language setting
% Replace `english' with e.g. `spanish' to change the document language
\usepackage[french]{babel}
\usepackage[fleqn]{amsmath} % Aligner les équations à gauche


% Set page size and margins
% Replace `letterpaper' with`a4paper' for UK/EU standard size
\usepackage[letterpaper,top=2cm,bottom=2cm,left=3cm,right=3cm,marginparwidth=1.75cm]{geometry}

% Useful packages
\usepackage{amsmath}
\usepackage{graphicx}
\usepackage{subcaption}
\usepackage[colorlinks=true, allcolors=blue]{hyperref}

\title{TD5}
\author{IPESUP - PC }
\date{6 Décembre 2023}

\begin{document}
\maketitle



\section{Fusible}

Un fusible est constitué par un fil conducteur cylindrique homogène, de section droite d’aire
$S$, de longueur utile $ L$, de masse volumique $\mu$ et de capacité thermique massique $c$. Il possède
une conductivité électrique $\gamma$ et une conductivité thermique $K$. Il est traversé par un courant
électrique continu d’intensité $I$. Ce fil est enfermé dans une capsule remplie d’une substance
assurant une isolation thermique et électrique parfaite. Les températures en $x =0 $ et $ x =L$
sont imposées et égales à la température $T_0$ du milieu ambiant.
Pour les applications numériques, on prendra les valeurs suivantes, données dans le système
international d’unités (SD) : $K = 65$ SI ; $\gamma$= 1,2 x 10 SI ; $c$ = 460 SI ; $\mu$ =$ 2,7 \times 10^3 kg.m^{-3}$ ;
$T_0 =290 K$; $L=2,5 cm$. 

\begin{enumerate}
    \item Montrer que la résistance d'un   conducteur cylindrique de conductivité $K$ de longueur
$L$, de section $S$, parcouru par un courant $I$ uniformément réparti et parallèle à son axe, est
$R=\frac{L}{K S}$ On se place en régime permanent. 
\item Établir l’équation différentielle vérifiée par la température $T$. Donner l’expression littérale
de $T(x)$ et représenter graphiquement $T$ en fonction de $x$. 
\item Le matériau constituant le fil fond à $T_f = 390 K$. On veut fabriquer un fusible qui admet
une intensité maximale $I_{max} = 16 A$.
Préciser l’endroit de la rupture en cas de dépassement de $I_{max}$. Déterminer l’expression littérale de l’aire $S$ à prévoir. Faire l’application numérique. 
\item  On fixe $ I = 10 A$. Le fil a la section $S$ trouvée à la question précédente. Evaluer littéralement puis numériquement la puissance thermique $P_{th}(0)$ transférée par conduction en $x= 0$.
Préciser si cette puissance est reçue ou fournie par le fil. Même question pour la puissance
thermique $P_{th}(L)$ transférée en $x = L$. Quelle relation a-t-on entre $P_{th}(0)$, $P_{th}(L)$ et la puissance
électrique $P$ fournie à l’ensemble du fil ? Commenter. 

\end{enumerate}

\section{Durée de survie d'un plongeur }
Un plongeur est équipé de sa combinaison.
On note $T_e$, la température de l’eau environnante, uniforme et constante. La température initiale du plongeur est $T_{i0}$ = 37°C. Les pertes thermiques ont lieu au niveau de la peau et de la
combinaison. 
\begin{enumerate}
    \item Rappeler l’expression de la résistance thermique dans le cas d’un modèle unidimensionnel,
en fonction de la section $S$, de l’épaisseur $e$ et de la conductivité thermique $\gamma$.
\item On modélise les pertes par convection par un flux thermique surfacique $\Phi = Sh(T-T_e)$
Quelle résistance $R_c$ peut-on associer aux pertes par convection ? 
\item  On modèlise les pertes par rayonnement par un flux thermique surfacique $\Phi_r = \sigma (T^4 - T_e^4)$
où $\sigma$ est la constante de Stefan. On suppose $|T-T_e|<<T_e$,. Montrer que l’on peut associer aux
pertes par rayonnement une résistance thermique $R_r$ dont on donnera l’expression en fonction
de $\sigma$, $T_e$, et de la surface $S$ du système. 
\item  Quelle est alors la résistance thermique $R_T$ équivalente à l’ensemble en fonction de $R_c$, $R_r$, $R_{peau}$, et $R_{combi}$ ?
\item  Établir l’équation différentielle vérifiée par $T_i(r)$ sachant que la puissance thermique produite par le métabolisme humain est $q = 120 W$ et sa capacité thermique massique $c = 3.5 kJ. kg^{-1}.K^{-1}$
\item Pour $T_e$ = 17°C, au bout de combien de temps le plongeur est-il en hypothermie, c’est-à- dire que sa température corporelle est descendue à 35°C ? Pour l’application numérique, on prendra la masse du plongeur $ m = 75 kg$, la surface totale de la combinaison $S = 1,3 m^2$, $R_{peau}$ = $3,0 \times 10^{-2} K.W^{-1}$, $\sigma$ =$5,7 \times 10^{-8} W.m^{-2} . K^{-4} SI.$,  $h = 10  W.m^{-2}.K^{-1}$, l’épaisseur e de la combinaison e = 3 mm et la conductivité thermique de la combinaison $\lambda = 4,4 \times 10^{-2} W.m^{-1}.K^{-1}$



\end{enumerate}

\section{Datation de l'age de la Terre}

La Terre est assimilée à un milieu semi-infini occupant tout le demi-espace z > 0. On admet que la température ne dépend que de la profondeur z (comptée positivement) et du temps t. La planète a une conductivité $\lambda$ , une masse volumique $\rho$ et une capacité thermique massique $c$, toutes trois uniformes. On note $j_Q(z,t)$ la densité de courant thermique. 

\begin{enumerate}
    \item  Établir rapidement l’équation aux dérivées partielles vérifiée par  $j_Q(z,t)$ (équation (1)). À On notera $D$ la diffusivité thermique, $D=\frac{\lambda  } {\rho c}$. 
    \\
Au milieu du XIX° siècle, Lord Kelvin a imaginé que la Terre a été formée à une température élevée uniforme $T_0 $ au moment $t=0$. Instantanément sa surface a été soumise à une température $T_s$. Depuis ce temps là, la planète se refroidirait. Lord Kelvin a modélisé ce refroidissement pour en déduire l’âge de formation de la Terre. 
\item  Dans l’hypothèse de Lord Kelvin, quelle doit être la valeur de la densité de courant thermique en $z = 0$ lorsque $t$ tend vers zéro, et lorsqu’il tend vers l’infini ? Quelle doit être la valeur de la densité de courant thermique à une profondeur $z$ non nulle lorsque $t$ tend vers zéro, et lorsqu’il tend vers l’infini ?
\item On admet que la fonction $f(z,t)= -\frac{1}{\sqrt{D t }}exp ( \frac{-z^2}{4Dt}) $ est solution de l’équation (1).
Vérifier que la solution proposée par Lord Kelvin, $j_Q(z,t)= -\frac{A}{\sqrt{D t }}exp ( \frac{-z^2}{4Dt})$ où $t$ est le
temps écoulé depuis la formation de la Terre est bien la bonne.
Dessiner schématiquement la valeur absolue de la densité de courant thermique en fonction de la profondeur pour deux époques différentes. 
\item On suppose que $A = a(T_o - T_s)^{\alpha}\lambda ^{\beta} \rho ^{\gamma} c^{\delta}$, où $a, \alpha, \beta, \gamma$ et $\delta$, où
sont des exposants éventuellement nuls. Calculer  $ \alpha, \beta, \gamma$ et $\delta$ par analyse de l’homogénéité de la formule de Lord
Kelvin. 
\item On peut montrer que $a=\frac{1}{\sqrt{\pi}}$. Exprimer $\frac{\partial T}{\partial z} $ la valeur du gradient thermique en surface de la
Terre . Lord Kelvin a admis que $T_0 - T_s$ était de l’ordre de 1000 à 2000 K et que D est
proche de $10^{-6} m^2.s^{-1}$. L’augmentation de température avec la profondeur mesurée dans les
mines indiquait un gradient thermique proche de 30 K/km. Quel âge de la Terre Lord Kelvin
a-t-il déduit de son modèle ? 
\item Que pensez vous de l’estimation précédente de l’âge de la Terre ? Quel est le ou les ingrédients physiques que Lord Kelvin n’aurait pas dû négliger ? Pourquoi l’a-t-il (ou les a-t-il)
négligé(s) ?


\end{enumerate}
\end{document}



\section{Exercice 1}

\section{Exercice 2}

\section{Exercice 3}

\section{ Formulaire }

\end{document}