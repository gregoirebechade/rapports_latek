\documentclass{article}

% Language setting
% Replace `english' with e.g. `spanish' to change the document language
\usepackage[french]{babel}
\usepackage[fleqn]{amsmath} % Aligner les équations à gauche


% Set page size and margins
% Replace `letterpaper' with`a4paper' for UK/EU standard size
\usepackage[letterpaper,top=2cm,bottom=2cm,left=3cm,right=3cm,marginparwidth=1.75cm]{geometry}

% Useful packages

\usepackage{amsmath}
\usepackage{graphicx}
\usepackage{subcaption}
\usepackage[colorlinks=true, allcolors=blue]{hyperref}

\title{TD 15 }
\author{IPESUP - PC }
\date{20 mars 2024}

\begin{document}
\maketitle



\section{Exercice}

On rappelle la constante de Plank : $h = 6,63 \times 10^{-34} J.s$ et la constante de Boltzmann : $k = 1,38 \times 10^{-23} J.K^{-1}$.
\begin{enumerate}
  \item On considère un milieu contenant des atome spouvant être dans l'état fondamental ($N_1$ atomes par unité de volume)
   et dénergie $E_1$ et dans un état excité ($N_2$ atomes par unité de volume) et d'énergie $E_2$.
   Estimer le rapport $\frac{N_2}{N_1}$ à température ambiante et pour une radiation visible. 
   \item Rappeler les expressions des nombres de photons absorbés et ceux émis par émission induite pendant $dt$.
   On donne la loi de Plank: $u_\nu = \frac{8\pi h \nu^3}{c^3} \frac{1}{exp({\frac{h\nu}{kT}})-1}$, avec $u_\nu$ la densité d'énergie par unité de fréquence.
   \item Estimer la valeur de A pour une raie d'une lampe à valeur atomique. 
   \item On se place à l'équilibre thermique. Trouver une relation entre $N_1$,  $N_2$, $A$, $B$ et $u_\nu$.
   \item En déduire une expression de $\frac{A}{B}$ pour une fréquence donnée. 
   \item On considère dorénavant un faisceau se propageant selon l'axe $z$ et de section $S$. On note $n$ la densité volumique de photons de fréquence $\nu$. Exprimer la puissance du faisceau en fonction de $n$. 
   \item Déterminer la valeur de $n$ pout un faisceau de longueur $632 nm$ et de puissance $1 mW$ et de $1mm$ de diamètre. 
\end{enumerate}


\section{Centrale PC 2 2018}

Questions 20 à 28. 


\end{document}



\section{Exercice 1}

\section{Exercice 2}

\section{Exercice 3}

\section{ Formulaire }

\end{document}