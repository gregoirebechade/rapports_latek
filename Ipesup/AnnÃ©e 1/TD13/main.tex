\documentclass{article}

% Language setting
% Replace `english' with e.g. `spanish' to change the document language
\usepackage[french]{babel}
\usepackage[fleqn]{amsmath} % Aligner les équations à gauche


% Set page size and margins
% Replace `letterpaper' with`a4paper' for UK/EU standard size
\usepackage[letterpaper,top=2cm,bottom=2cm,left=3cm,right=3cm,marginparwidth=1.75cm]{geometry}

% Useful packages
\usepackage{amsmath}
\usepackage{amssymb}
\usepackage{graphicx}
\usepackage{subcaption}
\usepackage[colorlinks=true, allcolors=blue]{hyperref}

\title{TD 13 }
\author{IPESUP - PC }
\date{7 mars 2024}

\begin{document}
\maketitle



\section{Changement de milieu }

On considère deux cordes accrochées à une masse $m$ située en $x=0$.
Une onde se propage sur ces deux cordes selon les $x$ croissants. 
On note $\underline{\xi_i}(x,t) = \xi_{0 i} e^{i(\omega t - k_1 x)}$ l'onde incidente,
et $\underline{\xi_r}(x,t) = \xi_{0 r} e^{i(\omega t + k_1 x)}$ l'onde réfléchie, 
et $\underline{\xi_r}(x,t) = \xi_{0 t} e^{i(\omega t - k_2 x)}$ l'onde transmise.

On pose $r=\frac{\underline{\xi_{0r}}}{\underline{\xi_{0i}}}$ le coefficient de réflexion complexe et 
$t=\frac{\underline{\xi_{0t}}}{\underline{\xi_{0i}}}$ le coefficient de transmission complexe.
\begin{enumerate}
  \item En appliquant la continuité de $\xi$ en 0, trouver une relation entre $r$ et $t$.
  \item En appliquant le PFD à la masse $m$, trouver une autre relation entre $r$ et $t$. On précisera les hypothèses faites. 
  \item En déduire $r$ et $t$.
\end{enumerate}


\section{Déformations longitudinales d'un ressort}

On considère un ressort à spires non jointives dont une extrémité est accrochées à $0$ fixe, et l'autre extrémité est accrochée à une masse $M$. 
On note $L$, $m_0$ et $k_0$ sa longueur à vide, sa masse et sa raideur. 
Le mouvement se fait le long de l'axe $x$.
On note $\mu = \frac{m_0}{L}$ la masse linéique du ressort.
On s'intéresse à la tranche de ressort comprise entre $x$ et $x+dx$ au repos.
On note $\xi(x,t)$ l'écart d'un point à sa position initiale. 
\begin{enumerate}
  \item Montrer qualitativement que la raideur de ce ressort infinitésimal est $\frac{k_0 L }{dx}$.
  \item Montrer que la force exercée par la partie droite du ressort sur la partie gauche du ressort vaut $\vec{F}(x,t)=R \frac{\partial \xi}{\partial x}(x,t) \vec{u_x}$. Donner la valeur de R. 
  \item Etablir l'équation vérifiée par $\xi(x,t)$.
  \item On cherche des solutions sous la forme $\xi(x,t) = f(x) cos(\omega t + \phi)$. Déterminer l'équation vérifiée par $\omega$ et montrer graphiquement que $\omega$ peut tendre vers l'infini.
  \item Retrouver la pulsation d'un ressort sans masse. 
  \item  On va maintenant préciser le résultat de la question précédente. 
    \begin{enumerate}
    \item Ecrire l'équation d'onde avec les variables adimensionnées $x_a = x/L$ et $t_a = t / T$, pù $L$ est la longueur du ressort et $T$ une durée caractéristique de la propagation des ondes. 
    \item On se place dans le cadre $L<<cT$. Montrer que l'équation d'onde devient $ \frac{\partial ^2 \xi}{\partial x ^2} \simeq 0 $. 
    \item Déterminer entièrement $\xi(x,t)$.
    \item Justifier la phrase suivante :  "Dans le cadre $L<<cT$, le ressort a une masse apparente de $m_0/3$".
    \end{enumerate}    
\end {enumerate}


\section{Chaîne de pendules}

On considère une chaîne de pendules simples, de masse $m$ et de longueur $l$. Ceux-ci sont régulièrement
espacés sur un fil de torsion horizontal de constante $C=\frac{\Gamma} {a}$. 
La distance entre deux pendules vaut $a$. 
On note $\theta_n(t)$ l'angle que fait le n-ième pendule avec la verticale.
\begin{enumerate}
  \item Exprimer le couple exercé par le pendule $n$ sur le fil situé à droite de ce dernier. 
  \item Ecrire l'équation du mouvement du pendule $n$.
  \item Dans la limite $  a \rightarrow 0$, déterminer l'équation vérifiée par $\theta (x,t)$. Montrer qu'il n'y a propagation que pour des pulsations supérieures à une pulsation de coupure $\omega_c$ que l'on précisera. 
\end{enumerate}

\section{Durée de chute}
On considère uen corde de longueur $l$ et de masse linéique $\mu$, suspendue par une de ses extrémités au plafond. 
A l'instant $t=0$, on amène l'autre extrémité de la corde au niveau de la première, puis on la lâche.
On suppose que le mouvement se fait sans dissipation de chaleur.
Calculer la durée de chute de la corde et la comparer avec la chute libre. 
Indication : $\int_{0}^{1}\sqrt{\frac{1-x}{x(2-x)} } dx \simeq 1.2$
% symbole pour ~



\end{document}