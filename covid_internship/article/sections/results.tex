\section*{Results}


\subsection{The data}

The models were trained and tested on many different points of 324 different pandemics generated with Covasim (\cite*{kerr2021covasim}), a complex agent-based model. 
Four parameters that maximized the diversity of the set of pandemics generated were selected (see \ref*{sec:generating_divserse_pandemics}). 
Those parameters were varied on the three values 0.5, 1 and 2. 
These values correspond to scaling factors of the default values of the model. 
For each pandemic, the transmission probabilities were time dependant and followed one of the four different mobility patterns (see \ref*{fig:mobilities}). 
Thanks to those 4 parameters, each varied on three different values and of the four different mobilities, we had 324 simulations of pandemics. 
We decided to focus on the number of hospitalized individuals as the target variable. 
Indeed, it is a key variable to monitor, as the hospital occupancy correspond to the capacity of the state to treat the patients. 


\subsection*{The models}

In this paper, we implemented 13 individual models and an ensemble model.
Each model was trained on the data from tye beggining of the pandemic and made predictions 7 or 14 daye ahead (on user's choice).
They output a point prediction and a set of confidence intervals on the prediction. 
Models of two types were implemented. 
The first type models were only trained on the time-serie of the number of hospitalized of the past few days. 
The second type of models was trained on the time series of the number of hospitalized, but also on the mobility data and the number of infected. 

\subsubsection{SIRH}

The SIRH (Suspectible, Infected, Recovered, Hospitalized) (\ref*{fig:sirh}) is a variation of the classical SIR model, with another compartment : the Hopsitalized compartment. 
The parameters of this model correspond to the rates of evolution from a compartment to another one. 
$\beta$ correspond to the transmission rate, $\gamma_i$ and $\gamma_h$ correspond to the recovery rate from both Infected and Hospitalized compartments, and $h$ correspond to the hospitalization rate. 
The value of $S, I, R$ and $H$ are linked through a system of differential equations (\ref{eq:sirh}). 
As the curve of the pandemic of a SIRH is deterministic, it is possible to generate it from the values of the parameters $\beta$, $\gamma_i$, $\gamma_h$ and $h$, and to compare it to the value of the number of hospitalized observed from the beginning of the pandemic.
The optimal parameters are found through minimization of the least square between the SIRH curve and the real curve. 
During the predicting phase, a 7 (or 14) days SIRH is run, with initial value estimated from the data and the fit of the training phase.
The confidence interval is estimated through linearization of the regression (see \ref{sec:ci})

Variations of the SIRH model were implemented, with $\gamma_i$ and $\gamma_h$ kept constant. 

A SIRH model of the second type was also implemented. 
Following the same idea as the first type SIRH, this model has a time varying transmission rate, which is a linear combination of the mobility : $\beta_t = a \times m_t + b$. 
The parameters to optimize are now $a, b, \gamma_i, \gamma_h$ and $h$ and both curve of the number of hospitalized and the number of infected are fitted to the data. 


\subsubsection{ARIMA}

The ARIMA is a model used for time series forcasting. 
It is the sum of an AR and a MA model (\ref*{eq:arima})
It is fitted to the data by maximizing the likelihood of the observed data. 

A VAR model, which is a multi-dimensional AR model was also implemented. 
It is also optimized through maximum likelihood. 

\subsubsection{Moving Average}

The moving average model is a baseline model that is used as a reference. 
It returns the value of the average pf the past seven days and a confidence interval based on the variance of the values of the past seven days. 


\subsubsection{Exponential regression}

The exponential regression is a model that fits an exponential regression ($x \rightarrow a e^{b x} + c$ ) to the values of the number of hospitaized. 
The values of $a, b$ and $c$ are found by minimizing the least square difference between the prediction and the real values. 
The confidence interval is computed through linearization of the regression (see \ref*{sec:ci}). 

An exponential regression of the second type was also computed. 
The number of hospitalized is then an exponential regression of both mobility data, number of infected and number of hospitalized shifted. 


\subsubsection{Regressors}

A Bayesian and a linear regressor were implemented.
They use the last 20 data points to predict the next one (see \ref*{sec:mlmodels}). 


\subsection*{Evaluation of the models}

The prediction

